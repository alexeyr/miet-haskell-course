\documentclass[11pt]{beamer}
\usepackage[utf8]{inputenc}
\usepackage[T1,T2A]{fontenc}
\usepackage[russian]{babel}
\usepackage{color}
\usepackage{calc}
\usepackage{graphicx}
\usepackage{epstopdf}
\usepackage{hyperref}
\hypersetup{unicode,colorlinks}
\usetheme[progressbar=head,numbering=fraction,block=fill]{metropolis}
\usepackage{minted}
\usepackage{dejavu}
%\usepackage{adjustbox}  % Позволяет сузить куски кода (или текст) ровно настолько, чтобы уместиться в слайд
\usepackage{csquotes}
\usepackage{upquote}

\usemintedstyle{solarized-light}
\newminted[haskell]{haskell}{
    escapeinside=!!,
    mathescape=true,
    texcomments=true,
    beameroverlays=true,
    autogobble=true,
    fontsize=\small,
    breaklines=false  % Лучше сам поставлю переносы на удобных местах
}
\newminted[haskellsmall]{haskell}{
    escapeinside=!!,
    mathescape=true,
    texcomments=true,
    beameroverlays=true,
    autogobble=true,
    fontsize=\footnotesize,
    breaklines=false
}
\newminted[haskelltiny]{haskell}{
    escapeinside=!!,
    mathescape=true,
    texcomments=true,
    beameroverlays=true,
    autogobble=true,
    fontsize=\scriptsize,
    breaklines=false
}
\newmintinline[haskinline]{haskell}{
    escapeinside=!!,
    mathescape=true,
    beameroverlays=true,
    breaklines=true
}
\newminted[ghci]{text}{
    autogobble=true,
    fontsize=\small,
    breaklines=false
}
\newminted[ghcismall]{text}{
    autogobble=true,
    fontsize=\footnotesize,
    breaklines=false
}
\newminted[ghcitiny]{text}{
    autogobble=true,
    fontsize=\scriptsize,
    breaklines=false
}
\newmintinline[ghcinline]{text}{
    breaklines=true
}

\newcommand{\hackage}[1]{\href{https://hackage.haskell.org/package/#1}{#1}}

\vfuzz=20pt  % позволяет тексту дойти до номера слайда

\author{Алексей Романов}
\subtitle{Функциональное программирование на Haskell}
%\logo{}
\institute{МИЭТ}
\subject{Функциональное программирование на Haskell}
%\setbeamercovered{transparent}
%\setbeamertemplate{navigation symbols}{}


\title{Лекция 6: работа со списками}

\begin{document}
\begin{frame}[plain]
\maketitle
\end{frame}

\begin{frame}[fragile]
\frametitle{Списки в Haskell}
\begin{itemize}
    \item Поговорим о списках подробнее.
    \item Это структура данных, которая постоянно встречается в программах.
    \item Но по поведению и характеристикам \emph{очень сильно} отличается от списков в Java и C\# (или \lstinline|std::vector| в C++).
\end{itemize}
\end{frame}

\begin{frame}[fragile]
\frametitle{Некоторые функции для работы со списками}
\begin{itemize}
    \item Первого порядка:
    \item TODO
    \item Второго порядка:
    \item TODO
\end{itemize}
\end{frame}

\begin{frame}[fragile]
\frametitle{Выделения списков (list comprehensions)}
\begin{itemize}
    \item К сожалению, \href{https://ru.wikipedia.org/wiki/Списковое_включение}{устоявшегося перевода этого термина нет}.
\end{itemize}
\end{frame}

\begin{frame}[fragile]
\frametitle{Свёртки списков}
\begin{itemize}
    \item TODO
\end{itemize}
\end{frame}

\begin{frame}[fragile]
\frametitle{Свёртки других типов данных}
\begin{itemize}
    \item Для \lstinline|Maybe|:
    \item TODO
\end{itemize}
\end{frame}

\end{document}