\documentclass[11pt]{beamer}
\usepackage[utf8]{inputenc}
\usepackage[T1,T2A]{fontenc}
\usepackage[russian]{babel}
\usepackage{color}
\usepackage{calc}
\usepackage{graphicx}
\usepackage{epstopdf}
\usepackage{hyperref}
\hypersetup{unicode,colorlinks}
\usetheme[progressbar=head,numbering=fraction,block=fill]{metropolis}
\usepackage{minted}
\usepackage{dejavu}
%\usepackage{adjustbox}  % Позволяет сузить куски кода (или текст) ровно настолько, чтобы уместиться в слайд
\usepackage{csquotes}
\usepackage{upquote}

\usemintedstyle{solarized-light}
\newminted[haskell]{haskell}{
    escapeinside=!!,
    mathescape=true,
    texcomments=true,
    beameroverlays=true,
    autogobble=true,
    fontsize=\small,
    breaklines=false  % Лучше сам поставлю переносы на удобных местах
}
\newminted[haskellsmall]{haskell}{
    escapeinside=!!,
    mathescape=true,
    texcomments=true,
    beameroverlays=true,
    autogobble=true,
    fontsize=\footnotesize,
    breaklines=false
}
\newminted[haskelltiny]{haskell}{
    escapeinside=!!,
    mathescape=true,
    texcomments=true,
    beameroverlays=true,
    autogobble=true,
    fontsize=\scriptsize,
    breaklines=false
}
\newmintinline[haskinline]{haskell}{
    escapeinside=!!,
    mathescape=true,
    beameroverlays=true,
    breaklines=true
}
\newminted[ghci]{text}{
    autogobble=true,
    fontsize=\small,
    breaklines=false
}
\newminted[ghcismall]{text}{
    autogobble=true,
    fontsize=\footnotesize,
    breaklines=false
}
\newminted[ghcitiny]{text}{
    autogobble=true,
    fontsize=\scriptsize,
    breaklines=false
}
\newmintinline[ghcinline]{text}{
    breaklines=true
}

\newcommand{\hackage}[1]{\href{https://hackage.haskell.org/package/#1}{#1}}

\vfuzz=20pt  % позволяет тексту дойти до номера слайда

\author{Алексей Романов}
\subtitle{Функциональное программирование на Haskell}
%\logo{}
\institute{МИЭТ}
\subject{Функциональное программирование на Haskell}
%\setbeamercovered{transparent}
%\setbeamertemplate{navigation symbols}{}


\title{Лекция 6: работа со списками}

\begin{document}
\begin{frame}[plain]
\maketitle
\end{frame}

\begin{frame}[fragile]
\frametitle{Списки в Haskell}
\begin{itemize}
    \item Поговорим о списках подробнее.
    \item Это структура данных, которая постоянно встречается в программах.
    \item Но по поведению и характеристикам \emph{очень сильно} отличается от списков в Java и C\# (или \lstinline|std::vector| в C++).
    \pause
    \item Списки в Haskell и множестве других языков, начиная с Lisp "--- односвязные.
    \pause
    \item То есть вместе с каждым элементом хранится ссылка на остальную часть списка.
    \item Вспомним определение:
\begin{lstlisting}
data [a] = [] | a : [a]
\end{lstlisting}
\end{itemize}
\end{frame}

\begin{frame}[fragile]
\frametitle{Некоторые функции первого порядка над\\ списками}
\begin{itemize}
    \item \lstinline|(++) :: [a] -> [a] -> [a]| \pause"--- объединение.
    \item \lstinline|(!!) :: [a] -> Int -> a| \pause"--- элемент по индексу.
    \item \lstinline|null :: a -> Bool| \pause"--- проверка на пустоту (лучше, чем \lstinline|length xs == 0|! Но сопоставление ещё лучше).
    \item \lstinline|elem :: Eq a => a -> [a] -> Bool| \pause"--- проверка на вхождение элемента.
    \item \lstinline|and, or :: [Bool] -> Bool| \pause"--- все ли элементы \lstinline|True| (или есть ли такой).
    \pause
    \item У части из них тип на самом деле более общий, но пока нам не нужен.
    \pause
    \item Все основные в \href{https://hackage.haskell.org/package/base-4.12.0.0/docs/Data-List.html}{Data.List}. \pause И ещё пакет \href{http://hackage.haskell.org/package/safe}{safe}.
\end{itemize}
\end{frame}

\begin{frame}[fragile]
\frametitle{Некоторые функции второго порядка над\\ списками}
\begin{itemize}
    \item \lstinline|map :: (a -> b) -> [a] -> [b]| \pause"--- применяет функцию ко всем элементам.
    \item \lstinline|filter :: (a -> Bool) -> [a] -> [a]| \pause"--- выбирает элементы, удовлетворяющие условию.
    \item \lstinline|concatMap :: (a -> [b]) -> [a] -> [b]| \pause"--- применяет функцию ко всем элементам и объединяет результаты.
    \item \lstinline|all, any :: (a -> Bool) -> [a] -> Bool| \pause"--- все ли элементы удовлетворяют условию (или есть ли такой).
    \pause
    \item[]
    \item Упражнение: реализуем какие-нибудь из них (рекурсивно и друг через друга).
\end{itemize}
\end{frame}

\begin{frame}[fragile]
\frametitle{Сложность функций над списками}
\begin{itemize}
    \item Базовые операции "--- конструирование списков и их разбор через \lstinline|:|.
    \pause
    \item Поэтому \lstinline|length| (а значит, и \lstinline|length xs == 0|) проходит весь список целиком, а \lstinline|null| нет.
    \pause
    \item \lstinline|xs !! n| "--- цена пропорциональна \pause\lstinline|n|.
    \pause
    \item Чем определяется цена \lstinline|xs ++ ys|? \pause Длиной \lstinline|xs|.
    \pause
    \item Добавление в конец списка \lstinline|xs ++ [y]| получается дорогим.
    \item В цикле может сделать линейный алгоритм квадратичным.
    \pause 
    \item Вместо добавления в конец часто лучше добавлять в начало, а потом развернуть.
    \pause
    \item Всё это осложняется ленивостью, о ней позже.
\end{itemize}
\end{frame}

\begin{frame}[fragile]
\frametitle{Выделения списков (list comprehensions)}
\begin{itemize}
    \item Есть удобный способ записи списков, похожий на выражения для множеств.
    \item[] \lstinline![x*x | x <- [1..5], even x] == !\pause \lstinline|[4, 16]|.
    \item Как \(\left\{x^2\,\middle|\,x \in \{1, \ldots, 5\},\,even(x)\right\}\).
    \item \lstinline!x <- [1..5]! "--- \emph{генератор}, \lstinline|even x| "--- \emph{фильтр}.
    \pause
    \item Тех и других может быть много.
    \item Генераторы могут использовать переменные, введённые в предыдущих. Пример:
\begin{lstlisting}[basicstyle=\ttfamily\footnotesize]
[(x, y) | x <- [1..5], y <- [1..x], odd (x - y)] == !\pause![(3, 2), (5, 2), (5, 4)]
\end{lstlisting}
    \item Эти выражения преобразуются в комбинации \lstinline|concatMap|, \lstinline|map| и \lstinline|filter|:
\begin{lstlisting}[basicstyle=\ttfamily\footnotesize]
[x*x | x <- [1..5], even x] == !\pause!
    map (\x -> x*x) (filter even [1..5])
\end{lstlisting}
\end{itemize}
\end{frame}

\begin{frame}[fragile]
\frametitle{Свёртки списков}
\begin{itemize}
    \item 
\begin{lstlisting}
foldr(') :: (a -> b -> b) -> b -> [a] -> b
foldr f z [x1, x2, ..., xn] == 
  x1 `f` (x2 `f` ... (xn `f` z)...)
foldl(') :: (b -> a -> b) -> b -> [a] -> b
foldl f z [x1, x2, ..., xn] == 
  (...((z `f` x1) `f` x2) `f`...) `f` xn
\end{lstlisting}
    \item Функции \lstinline|sum|, \lstinline|length| и т.д. можно реализовать через них (TODO)
\end{itemize}
\end{frame}

\begin{frame}[fragile]
\frametitle{Свёртки других типов данных}
\begin{itemize}
    \item Сравним объявление списка с типом \lstinline|foldr|:
\begin{lstlisting}
data [a] = a : [a]       | []
foldr ::  (a -> b -> b) -> b -> [a] -> b
\end{lstlisting}
    \item Видим, что у нас по одному аргументу для каждого конструктора: каждый принимает аргументы типов полей этого конструктора (а если там \lstinline|[a]|, то \lstinline|b|) и возвращает \lstinline|b|. 
    \item Последний аргумент "--- разбираемый список.
    \item Для \lstinline|Maybe|:
\begin{lstlisting}[basicstyle=\ttfamily\footnotesize]
data Maybe a = Nothing | Just a
foldMaybe ::   ?      -> ?        -> ?       -> b !\pause!
foldMaybe ::   b      -> (a -> b) -> Maybe b -> b
\end{lstlisting}
    \pause
    \item Это \lstinline|Data.Maybe.maybe| с обратным порядком аргументов. (TODO Bool и деревья)
\end{itemize}
\end{frame}

\end{document}