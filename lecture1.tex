\documentclass[10pt]{beamer}
\usepackage[utf8]{inputenc}
\usepackage[T1,T2A]{fontenc}
\usepackage[russian]{babel}
\usepackage{color}
\usepackage{calc}
\usepackage{graphicx}
\usepackage{epstopdf}
\usepackage{hyperref}
\hypersetup{unicode,colorlinks}
\usetheme[progressbar=head,numbering=fraction,block=fill]{metropolis}
\usepackage{minted}
\usepackage{dejavu}
%\usepackage{adjustbox}  % Позволяет сузить куски кода (или текст) ровно настолько, чтобы уместиться в слайд
\usepackage{csquotes}
\usepackage{upquote}

\usemintedstyle{solarized-light}
\newminted[haskell]{haskell}{
    escapeinside=!!,
    mathescape=true,
    texcomments=true,
    beameroverlays=true,
    autogobble=true,
    fontsize=\small,
    breaklines=false  % Лучше сам поставлю переносы на удобных местах
}
\newminted[haskellsmall]{haskell}{
    escapeinside=!!,
    mathescape=true,
    texcomments=true,
    beameroverlays=true,
    autogobble=true,
    fontsize=\footnotesize,
    breaklines=false
}
\newminted[haskelltiny]{haskell}{
    escapeinside=!!,
    mathescape=true,
    texcomments=true,
    beameroverlays=true,
    autogobble=true,
    fontsize=\scriptsize,
    breaklines=false
}
\newmintinline[haskinline]{haskell}{
    escapeinside=!!,
    mathescape=true,
    beameroverlays=true,
    breaklines=true
}
\newminted[ghci]{text}{
    autogobble=true,
    fontsize=\small,
    breaklines=false
}
\newminted[ghcismall]{text}{
    autogobble=true,
    fontsize=\footnotesize,
    breaklines=false
}
\newminted[ghcitiny]{text}{
    autogobble=true,
    fontsize=\scriptsize,
    breaklines=false
}
\newmintinline[ghcinline]{text}{
    breaklines=true
}

\newcommand{\hackage}[1]{\href{https://hackage.haskell.org/package/#1}{#1}}

\vfuzz=20pt  % позволяет тексту дойти до номера слайда

\author{Алексей Романов}
\subtitle{Функциональное программирование на Haskell}
%\logo{}
\institute{МИЭТ}
\subject{Функциональное программирование на Haskell}
%\setbeamercovered{transparent}
%\setbeamertemplate{navigation symbols}{}


\title{Лекция 1: введение и основы синтаксиса}

\begin{document}
\begin{frame}[plain]
  \maketitle
\end{frame}

\begin{frame}
  \frametitle{Организация курса}
  \begin{itemize}
    \item 8 лекций
    \item 4 лабораторных
    \item Итоговый проект
  \end{itemize}
\end{frame}

\begin{frame}
  \frametitle{Парадигмы программирования}
  \begin{itemize}
    \item Что такое парадигма?
          \pause
          \begin{quote}
            \enquote{Совокупность идей и понятий, определяющих стиль написания компьютерных программ.} (Wikipedia)
          \end{quote}
          \pause
    \item Основные парадигмы:
          \pause
          \begin{itemize}
            \item Структурное программирование
            \item Процедурное программирование
            \item \textbf{Функциональное программирование}
            \item Логическое программирование
            \item Объектно-ориентированное программирование
          \end{itemize}
          \pause
    \item В парадигме важно не только то, что используется, но то, использование чего не допускается или минимизируется.
          \pause
    \item Например, \haskinline!goto! в структурном программировании, глобальные переменные в ООП.
  \end{itemize}
\end{frame}

\begin{frame}
  \includegraphics[trim={0 0 144mm -2mm},clip,width=\textwidth,height=\textheight,keepaspectratio]{paradigmsDIAGRAMeng108.pdf}
  \footnotesize  \href{https://www.info.ucl.ac.be/~pvr/paradigms.html}{Peter Van Roy, "Programming Paradigms for Dummies: What Every Programmer Should Know", 2009}
\end{frame}

\begin{frame}
  \frametitle{Функциональное программирование}
  \begin{itemize}
    \item Значения лучше переменных.
          \begin{itemize}
            \item Переменная даёт имя значению или функции, а не адресу в памяти.
            \item Переменные неизменяемы.
            \item Типы данных неизменяемы.
          \end{itemize}
    \item Выражения лучше инструкций.
          \begin{itemize}
            \item Аналоги \haskinline|if|, \haskinline|try-catch| и т.д. "--- выражения.
          \end{itemize}
    \item Функции как в математике (следующий слайд)
  \end{itemize}
\end{frame}


\begin{frame}
  \frametitle{Функциональное программирование}
  \begin{itemize}
    \item Функции как в математике\pause
          \begin{itemize}
            \item Чистые функции: аргументу соответствует результат, а всё прочее от лукавого.
                  \begin{itemize}
                    \item Нет побочных эффектов (ввода-вывода, обращения к внешней памяти, не связанной с аргументом, и т.д.)
                    \item При одинаковых аргументах результаты такой функции одинаковы
                  \end{itemize}
            \item Функции являются значениями (функции первого класса)
            \item Функции часто принимают и возвращают функции (функции высших порядков)
          \end{itemize}
          \pause
    \item Опора на математические теории: лямбда-исчисление, теория типов, теория категорий
  \end{itemize}
\end{frame}

\begin{frame}
  \frametitle{Языки ФП}
  \begin{itemize}
    \item семейство Lisp: первый ФП-язык и один из первых языков высокого уровня вообще
    \item Erlang и Elixir: упор на многозадачность (модель акторов), надёжность
    \item Scala, Kotlin, F\#: гибриды с ООП для JVM и для CLR
    \item Purescript, Elm, Ur/Web: для веба
    \item Семейство ML: OCaml, SML, F\#
          \pause
    \item \textbf{Haskell:}
          \begin{itemize}
            \item Чисто функциональный
                  \pause
            \item Строго статически типизированный (с очень мощной и выразительной системой типов)
                  \pause
            \item Ленивый
          \end{itemize}
  \end{itemize}
\end{frame}

\begin{frame}[fragile]
  \frametitle{Язык Haskell: начало}
  \begin{itemize}
    \item Установите ghcup (\url{https://www.haskell.org/ghcup/})
    \item Запустите GHCi
    \item Это оболочка или REPL (Read-Eval-Print loop)
          \begin{itemize}
            \item Read: Вы вводите выражения Haskell (и команды GHCi)
            \item Eval: GHCi вычисляет результат
            \item Print: и выводит его на экран
          \end{itemize}
    \item Пример:
          \begin{ghci}
            GHCi, version 9.2.5: http://www.haskell.org/ghc/  :? for help
          \end{ghci}
          \begin{haskell}
            ghci> 2 + 2
            4
            ghci> :t True -- команда GHCi
            True :: Bool
          \end{haskell}
  \end{itemize}
\end{frame}

\begin{frame}[fragile]
  \frametitle{Язык Haskell: начало}
  \begin{itemize}
    \item \haskinline|2 + 2|, \haskinline|True| "--- выражения
    \item \haskinline|4|, \haskinline|True| "--- значения
    \item \haskinline|Bool| "--- тип
          \pause
    \item Значение "--- \enquote{вычисленное до конца} выражение.
    \item Тип (статический) "--- множество значений и выражений, построенное по таким правилам, что компилятор может определить типы и проверить отсутствие ошибок в них без запуска программы.
    \item От типа зависит то, какие операции допустимы:
          \begin{haskell}
            ghci> True + False
          \end{haskell}
          \begin{ghci}
            <interactive>:12:1: error:
            No instance for (Num Bool) arising from a use of '+'
            In the expression: True + False
            In an equation for 'it': it = True + False
          \end{ghci}
    \item Это ошибка компиляции, а не выполнения.
  \end{itemize}
\end{frame}

\begin{frame}[fragile]
  \frametitle{Вызов функций}
  \begin{itemize}
    \item Вызов (применение) функции пишется без скобок: \haskinline|f x|, \haskinline|foo x y|.
    \item Скобки используются, когда аргументы "--- сложные выражения: \haskinline|f (g x)|
    \item И внутри сложных выражений вообще.
    \item Бинарные операторы (как \haskinline|+|) это просто функции с именем из символов вместо букв и цифр.
          \begin{itemize}
            \item Можно писать их префиксно, заключив в скобки: \haskinline[breaklines=false]|(+) 2 2|.
            \item А любую функцию двух аргументов с алфавитным именем можно писать инфиксно между обратными апострофами:
                  \haskinline|4 `div` 2|.
            \item Единственный небинарный оператор "--- унарный \haskinline|-|.
          \end{itemize}
    \item Названия переменных и функций начинаются со строчной буквы (кроме операторов).
  \end{itemize}
\end{frame}

\begin{frame}[fragile]
  \frametitle{Определение функций и переменных}
  \begin{itemize}
    \item Определение переменной выглядит как в математике, даже без ключевых слов:
          \begin{haskell}
            название = значение
          \end{haskell}
    \item Определение функции почти такое же:
          \begin{haskell}
            название параметр1 параметр2 = значение
          \end{haskell}
    \item Тело функции это не блок, а одно выражение (но сколь угодно сложное).
          \begin{haskell}
            ghci> x = sin pi
            ghci> x !\pause!
            1.2246063538223773e-16
            ghci> square x = x * x
            ghci> square 2
            4
          \end{haskell}
  \end{itemize}
\end{frame}

\begin{frame}[fragile]
  \frametitle{Базовые типы}
  \begin{itemize}
    \item Названия типов всегда с заглавной буквы.
    \item \haskinline|Bool|: логические значения \haskinline|True| и \haskinline|False|.
    \item Целые числа:
          \begin{itemize}
            \item \haskinline|Integer|: неограниченные (кроме размера памяти);
            \item \haskinline|Int|: машинные\footnote{по cтандарту минимум 30 бит, но в GHC именно 32 или 64 бита}, \haskinline|Word|: машинные без знака;
            \item \haskinline|Data.{Int.Int/Word.Word}{8/16/32/64}|: фиксированного размера в битах, со знаком и без.
          \end{itemize}
    \item \haskinline|Float| и \haskinline|Double|: 32- и 64-битные числа с плавающей точкой, по стандарту IEEE-754.
    \item \haskinline|Character|: символы Unicode.
    \item \haskinline|()|: \enquote{Единичный тип} (unit) с единственным значением \haskinline|()|.
  \end{itemize}
\end{frame}

\begin{frame}[fragile]
  \frametitle{Тип функций и сигнатуры}
  \begin{itemize}
    \item Типы функций записываются через \haskinline|->|. Например, \haskinline|Int -> Char| это тип функции из \haskinline|Int| в \haskinline|Char|.

    \item Для нескольких аргументов это выглядит как \pause\haskinline|Bool -> Bool -> Bool|.

    \item \haskinline|::| читается как \enquote{имеет тип}.

    \item Запись \haskinline|выражение :: тип| "--- \enquote{сигнатура типа}.

    \item При объявлении экспортируемой функции или переменной  сигнатура обычно указывается явно:
          \begin{haskell}
            foo :: Int -> Char
            foo x = ...
          \end{haskell}
    \item Компилятор обычно может вывести типы сам, но это защищает от \emph{непреднамеренного} изменения.
  \end{itemize}
\end{frame}

\begin{frame}[fragile]
  \frametitle{Арифметика}
  \begin{itemize}
    \item Это упрощённая версия.
    \item Полное объяснение требует понятия, которое будет введено позже.
    \item Например, команды \haskinline|:type 1| или \haskinline|:type (+)| дадут тип, который понимать пока не требуется.
    \item То же относится к ошибкам вроде \ghcinline[breaklines=false]|No instance for (Num Bool)...|\\ на более раннем слайде.
    \item Пока достаточно понимать, что есть несколько числовых типов.
    \item Они делятся на целочисленные (\haskinline|Int|, \haskinline|Integer|) и дробные (\haskinline|Float|, \haskinline|Double|).
  \end{itemize}
\end{frame}

\begin{frame}[fragile]
  \frametitle{Числовые литералы}
  \begin{itemize}
    \item Числовые литералы выглядят, как в других языках: \haskinline|0|, \haskinline|1.5|, \haskinline|1.2E-1|, \haskinline|0xDEADBEEF|.
    \item Целочисленные литералы могут иметь любой числовой тип, а дробные любой дробный.
    \item Но это относится \emph{только} к литералам.
    \item Неявного приведения (например, \haskinline|Int| в \haskinline|Double|) в Haskell \emph{нет}.
  \end{itemize}
\end{frame}

\begin{frame}[fragile]
  \frametitle{Приведение числовых типов}
  \begin{itemize}
    \item Используйте \haskinline|fromIntegral| для приведения из любого целочисленного типа в любой числовой. Тип-цель можно указать явно:
          \begin{haskell}
            ghci> let {x :: Integer; x = 2}
            ghci> x :: Double
          \end{haskell}
          \begin{ghci}
            <interactive>:22:1: error:
            Couldn't match expected type 'Double' with actual type 'Integer' ...
          \end{ghci}
          \begin{haskell}
            ghci> fromIntegral x :: Double
            2.0
          \end{haskell}
    \item А может выводиться из контекста:
          \begin{haskell}
            ghci> :t fromIntegral 2 / (4 :: Double)
            fromIntegral 2 / (4 :: Double) :: Double
          \end{haskell}
  \end{itemize}
\end{frame}

\begin{frame}[fragile]
  \frametitle{Приведение числовых типов}
  \begin{itemize}
    \item \haskinline|toInteger| переводит любой целочисленный тип \\в \haskinline|Integer|.
    \item \haskinline|fromInteger| "--- наоборот. Если аргумент слишком велик, возвращает значение по модулю:
          \begin{haskell}
            ghci> fromInteger (2^64) :: Int
            0
          \end{haskell}
    \item \haskinline|toRational| и \haskinline|fromRational| "--- аналогично для \haskinline|Rational| и дробных типов.
    \item \haskinline|ceiling|, \haskinline|floor|, \haskinline|truncate| и \haskinline|round| "--- из дробных типов в целочисленные.
  \end{itemize}
\end{frame}

\begin{frame}[fragile]
  \frametitle{Арифметические операции}
  \begin{itemize}
    \item Операции \haskinline|+|, \haskinline|-|, \haskinline|*| "--- как обычно (унарного \haskinline|+| нет).
    \item \haskinline|/| "--- деление \emph{дробных} чисел.
          \begin{itemize}
            \item Использование \haskinline|/| для целых чисел даст ошибку \ghcinline|No instance... arising from a use of '/'|. Нужно сначала использовать \haskinline|fromIntegral|.
          \end{itemize}
    \item \haskinline|div| "--- деление нацело
    \item \haskinline|mod| "--- остаток
    \item
          \haskinline|quot| и \haskinline|rem| тоже, но отличаются от них поведением на отрицательных числах.
    \item \haskinline|^| "--- возведение любого числа в неотрицательную  целую степень.
    \item \haskinline|^^| "--- дробного в любую целую.
    \item \haskinline|**| "--- дробного в степень того же типа.
  \end{itemize}
\end{frame}

\begin{frame}[fragile]
  \frametitle{Операции сравнения и\\логические операции}
  \begin{itemize}
    \item Большинство операций сравнения выглядят как обычно: \haskinline|==|, \haskinline|>|, \haskinline|<|, \haskinline|>=|, \haskinline|<=|.
    \item Но $\neq$ обозначается как \haskinline|/=|.
    \item Функция \haskinline|compare| возвращает \haskinline|Ordering|: тип с тремя значениями \haskinline|LT|, \haskinline|EQ| и \haskinline|GT|.
    \item Есть функции \haskinline|min| и \haskinline|max|.
    \item[]
    \item \enquote{и} это \haskinline|&&|, а \enquote{или} "--- \haskinline!||!, как обычно.
    \item \enquote{не} "--- \haskinline|not|.
  \end{itemize}
\end{frame}

%\begin{frame}
%    \href{run:./lecture2.pdf}{\beamergotobutton{Следующая лекция}}
%\end{frame}
\end{document}
