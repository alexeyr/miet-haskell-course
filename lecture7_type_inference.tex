\documentclass[11pt]{beamer}
\usepackage[utf8]{inputenc}
\usepackage[T1,T2A]{fontenc}
\usepackage[russian]{babel}
\usepackage{color}
\usepackage{calc}
\usepackage{graphicx}
\usepackage{epstopdf}
\usepackage{hyperref}
\hypersetup{unicode,colorlinks}
\usetheme[progressbar=head,numbering=fraction,block=fill]{metropolis}
\usepackage{minted}
\usepackage{dejavu}
%\usepackage{adjustbox}  % Позволяет сузить куски кода (или текст) ровно настолько, чтобы уместиться в слайд
\usepackage{csquotes}
\usepackage{upquote}

\usemintedstyle{solarized-light}
\newminted[haskell]{haskell}{
    escapeinside=!!,
    mathescape=true,
    texcomments=true,
    beameroverlays=true,
    autogobble=true,
    fontsize=\small,
    breaklines=false  % Лучше сам поставлю переносы на удобных местах
}
\newminted[haskellsmall]{haskell}{
    escapeinside=!!,
    mathescape=true,
    texcomments=true,
    beameroverlays=true,
    autogobble=true,
    fontsize=\footnotesize,
    breaklines=false
}
\newminted[haskelltiny]{haskell}{
    escapeinside=!!,
    mathescape=true,
    texcomments=true,
    beameroverlays=true,
    autogobble=true,
    fontsize=\scriptsize,
    breaklines=false
}
\newmintinline[haskinline]{haskell}{
    escapeinside=!!,
    mathescape=true,
    beameroverlays=true,
    breaklines=true
}
\newminted[ghci]{text}{
    autogobble=true,
    fontsize=\small,
    breaklines=false
}
\newminted[ghcismall]{text}{
    autogobble=true,
    fontsize=\footnotesize,
    breaklines=false
}
\newminted[ghcitiny]{text}{
    autogobble=true,
    fontsize=\scriptsize,
    breaklines=false
}
\newmintinline[ghcinline]{text}{
    breaklines=true
}

\newcommand{\hackage}[1]{\href{https://hackage.haskell.org/package/#1}{#1}}

\vfuzz=20pt  % позволяет тексту дойти до номера слайда

\author{Алексей Романов}
\subtitle{Функциональное программирование на Haskell}
%\logo{}
\institute{МИЭТ}
\subject{Функциональное программирование на Haskell}
%\setbeamercovered{transparent}
%\setbeamertemplate{navigation symbols}{}

\usepackage{syntax}

\title{Лекция 7: вывод типов}
\date{4 апреля 2018}

\begin{document}
\begin{frame}[plain]
\maketitle
\end{frame}

\begin{frame}[fragile]
\frametitle{Вывод типов в других языках}
\begin{itemize}
    \item Во многих ООП-языках сейчас есть вывод типов локальных переменных (иногда и возвращаемого типа):
    \hspace{-1em}
    \pause
    \begin{itemize}
        \item \lstinline!auto! в C++11
        \item \lstinline!var! в C\# 3.0 и Java 10
        \item Отсутствие явного типа в Kotlin, Scala
    \end{itemize}
    \pause
    \item Устроен в них всех похоже:
    \begin{itemize}
        \pause
        \item Тип аргументов функций задаётся явно.
        \pause
        \item Дальше типы протягиваются сверху вниз.
        \pause
        \item Тип локальных переменных/полей $=$ тип инициализатора (если нет явного).
        \pause
        \item Возвращаемый тип $=$ общий тип для всех \lstinline!return!.
    \end{itemize}
    \pause
    \item Отдельный вопрос: вывод параметров шаблонов (в C++)/генериков (в других языках).
    \pause
    \item Он сложнее, но появился раньше.
\end{itemize}
\end{frame}

\begin{frame}[fragile]
\frametitle{Вывод типов в системе Хиндли-Милнера}
\begin{itemize}
    \item В основе системы типов языка Haskell лежит система Хиндли-Милнера.
    \item Она изначально создана для вывода типов.
    \pause
    \item Даже без указания типов параметров.
    \pause
    \item Плюсы: 
    \begin{itemize}
        \pause
        \item Для любого данного выражения есть наиболее общий тип (если есть хоть какой-то).
        \pause
        \item Есть эффективные (для реальных программ) алгоритмы их нахождения.
    \end{itemize}
    \pause
    \item Не все расширения Haskell их сохраняют (но это один из критериев оценки расширений).
    \pause
    \item Минусы:
    \begin{itemize}
        \pause
        \item Сложнее понять, как работает.
        \pause
        \item Больше \enquote{дальнодействия}.
    \end{itemize}
    \pause
    \item Поэтому понимание работы иногда необходимо.
\end{itemize}
\end{frame}

\begin{frame}[fragile]
\frametitle{Постановка задачи}
\begin{itemize}
    \item Даны:
    \begin{itemize}
        \item Набор типов и конструкторов типов: \lstinline|Int|, \lstinline|Bool|, \lstinline|[]| и т.д.
        \item Среди них особую роль играет конструктор функций \lstinline|->|.
        \item Набор известных именованных функций (или констант) с их типами.
        \item Выражение, тип которого мы хотим найти (обычно определение функции).
    \end{itemize}
    \pause
    \item Нужно: 
    \begin{itemize}
        \item Определить наиболее общий тип для этого выражения.
        \pause
        \item Или найти объяснение, почему ему нельзя дать никакой тип.
    \end{itemize}
\end{itemize}
\end{frame}

\begin{frame}[fragile]
\frametitle{\lstinline|let| и $\lambda$}
\begin{itemize}
    \item Сравните \lstinline[mathescape]|let x = e$_1$ in e$_2$| и \lstinline[mathescape]|(\x -> e$_2$) e$_1$|.
    \pause
    \item Значения всегда одинаковы (для любых \lstinline[mathescape]|e$_1$| и \lstinline[mathescape]|e$_2$|).
    \pause
    \item Но есть разница при типизации: \pause \lstinline|x| в лямбде всегда мономорфна, а в \lstinline|let| может быть полиморфной.
    \pause
    \item Поэтому в нетипизированном и просто типизированном $\lambda$-исчислении \lstinline|let| обычно не вводится.
\end{itemize}
\end{frame}

\begin{frame}[fragile]
\frametitle{Виды выражений и типов}
\begin{itemize}
    \item В нашей системе 5 видов выражений:
    \pause
    \item Названия переменных (в том числе функций) $x$ 
    \item Применение функции к аргументу
    \item Лямбда-выражения
    \item \lstinline|let|-выражения    
    \item \lstinline|case|-выражения    
    \item[]
    \pause
    \item Типы $t$:
    \pause
    \item Базовые типы \lstinline|Int|, \ldots
    \item Переменные типов (греческие буквы $\alpha, \beta, \ldots$)
    \item Тип функций $t_1 \to t_2$
    \item[]
    \pause
    \item Полиморфные типы: $\forall \alpha_1 \ldots \alpha_k t$
    \pause
    \item $\forall$ только на внешнем уровне, под $\to$ его быть не может!
\end{itemize}
\end{frame}


\begin{frame}[fragile]
\frametitle{Унификация}
\begin{itemize}
    \item Нам понадобятся понятия подстановки и унификации. В общем случае:
    \pause 
    \item Пусть даны два выражения (терма) какого-то формального языка (или несколько пар). 
    \item Они могут содержать переменные.
    \pause
    \item Подстановка переменных сопоставляет некоторым переменным выражения (которые тоже могут содержать переменные).
    \pause
    \item Мы хотим знать, можно ли сделать такую подстановку, чтобы термы стали одинаковыми.
    \pause
    \item Это достаточно широко применимое понятие.
    \pause
    \item Разрешимость задачи унификации и свойства решений зависят от структуры языка. 
\end{itemize}
\end{frame}

\begin{frame}[fragile]
\frametitle{Унификация типов}
\begin{itemize}
    \item Для вывода типов в системе Хиндли-Милнера случай один из самых простых: унификация первого порядка.
    \item Подстановки $\sigma$ сопоставляют переменным типов $\alpha, \beta, \ldots$ \emph{мономорфные} типы (без $\forall$). Например,
    \[ \sigma=\{\alpha \mapsto \mathtt{Int}, \beta \mapsto \mathtt{[}\gamma\mathtt{]} \} \]
    \item Подстановку можно применить к типу:
    \[ \sigma(\mathtt{Maybe}~\beta)=\mathtt{Maybe}~\mathtt{[}\gamma\mathtt{]} \]
    \item Композиция подстановок "--- подстановка.
    \item Мономорфные типы равны, если они совпадают синтаксически. 
    \pause
    \begin{itemize}
        \item Т.е. синонимы типов должны быть уже раскрыты!
    \end{itemize}
\end{itemize}
\end{frame}

\begin{frame}[fragile]
\frametitle{Унификация типов}
\begin{itemize}
    \item Тип \lstinline[mathescape]|($\alpha$, $\alpha$)| более общий, чем \lstinline[mathescape]|(Int, Int)|, так как  \lstinline[mathescape]|$\exists \sigma~\sigma($($\alpha$, $\alpha$)$)=$(Int, Int)|.
    \item Второй "--- частный случай первого.
    \pause
    \item \lstinline[mathescape]|($\alpha$, $\beta$)| более общий, чем \lstinline[mathescape]|($\alpha$, $\alpha$)|.
    \pause
    \item Вопрос: когда два типа более общи друг друга?
    \pause
    \item Когда они отличаются только названиями переменных.
    \pause
    \item Можем рассмотреть задачу унификации \lstinline[mathescape]|($\alpha$, $\beta$)$=$($\beta$, Int)|.
    \pause
    \item $\{ \alpha \mapsto \mathtt{Int}, \beta \mapsto \mathtt{Int} \}$ это одно из её решений.
    \pause
    \item Наиболее общее (все остальные "--- его частные случаи).
\end{itemize}
\end{frame}

\begin{frame}[fragile]
\frametitle{Алгоритм унификации}
\begin{itemize}
    \item Вообще, у каждой задачи унификации типов есть наиболее общее решение (или нет вообще). 
    \item Оно находится следующим алгоритмом:
    \pause
    \item На каждом шаге есть система уравнений. Выбираем любое из них и упрощаем.
    \item Закончим, когда не останется упрощаемых уравнений (оставшиеся опишут подстановку).
    \item Правило упрощения зависит от вида рассмотренного уравнения:
\end{itemize}
\end{frame}

\begin{frame}[fragile]
\frametitle{Алгоритм унификации: правила}
\begin{itemize}
    \item $C~t_1~t_2~\ldots~t_k = C~s_1~s_2~\ldots~s_k$ (одинаковые конструкторы): \pause заменяется на $t_1 = s_1, t_2 = s_2, \ldots$
    \pause
    \item $C~\ldots = D~\ldots$ (разные конструкторы): \pause унификатора не существует.
    \pause
    \item $x=x$: \pause оно удаляется, переходим к следующему (можно обобщить и на $t=t$).
    \pause
    \item $x=t$ или $t=x$: два случая. \pause
    \begin{itemize}
        \item Если в $t$ нет переменной $x$: \pause во всех остальных уравнениях делаем замену $x \mapsto t$, переходим к следующему.
        \item Если есть\pause, то унификатора не существует (иначе результат будет бесконечным)!
    \end{itemize}
\end{itemize}
\end{frame}

\newcommand{\fresh}{\mathord{\mathit{fresh}}()}
\newcommand{\J}{\mathcal{J}}
\begin{frame}[fragile]
\frametitle{Алгоритм $\J$ вывода типов}
\begin{itemize}
    \item $\Gamma$ "--- окружение типов (набор переменных с их типами).
    \item $\J(\Gamma; e)$ возвращает наиболее общий тип для $e$ с окружением $\Gamma$ или выдаёт ошибку.
    \pause
    \vspace{1em}
    \item $\phi$ "--- глобальная переменная (только для простоты объяснения), содержащая уже сделанные подстановки.
    \item Функция $\fresh$ возвращает свежую переменную типа (т.е. такую, которой ещё нигде не было).
    \item Функция $ftv$ возвращает свободные переменные типов в аргументе.
    \item Функция $unify$ решает задачу унификации.
\end{itemize}
\end{frame}

\begin{frame}[fragile]
\frametitle{Алгоритм $\J$: приведение $e$ к стандартному виду}
\begin{itemize}
    \item Переименуем связанные переменные с одинаковыми именами.
    \item Заменим \lstinline|where| на \lstinline|let|.
    \item \lstinline|if| и все сопоставления с образцом в \lstinline|case|.
\end{itemize}
\end{frame}

\begin{frame}[fragile]
\frametitle{Алгоритм $\J$: рекурсия по определению $e$}
\begin{itemize}
    \item Переменная $x$.
    \pause
    \begin{itemize}
        \item Тип берётся из окружения (он всегда там есть). 
        \item Если он полиморфный, то переменные под кванторами заменяются на свежие (новые).
    \end{itemize}
    \pause
    \item[]
    \item Применение функции $e_1~e_2$.
    \pause
    \begin{itemize}
        \item Выводим тип $t_1$ для $e_1$.
        \item Выводим тип $t_2$ для $e_2$.
        \item $t_1$ должен иметь вид $t_2 \to \alpha$ \\(где $\alpha$ "--- свежая переменная).
        \\Унифицируем их.
        \item Результат (тип $e_1~e_2$): \pause $\alpha$ (после подстановок).
    \end{itemize}
\end{itemize}
\end{frame}

\begin{frame}[fragile]
\frametitle{Алгоритм $\J$: рекурсия по определению $e$}
\begin{itemize}
    \item Лямбда-выражение $\lambda x.e_1$.
    \pause
    \begin{itemize}
        \item Тип $x$ "--- свежая переменная $\alpha$. \\ $x : \alpha$ добавляется в окружение (на время этого вывода).
        \item Выводится тип $t_1$ для $e_1$.
        \item Результат: \pause $\alpha \to t_1$.
    \end{itemize}
    \pause
    \item[]
    \item \lstinline|let|-выражение \lstinline[mathescape]|let $x$ = $e_1$ in $e_2$|
    \pause
    \begin{itemize}
        \item Выводим тип $t_1$ для $e_1$.
        \item Тип $x$ получается навешиванием $\forall$ на свободные переменные $t_1$, не входящие в окружение.
        \\ Добавляем его в окружение \pause (тоже на время этого вывода).
        \item Выводим тип $t_2$ для $e_2$.
        \item Результат: \pause $t_2$.
    \end{itemize}
\end{itemize}
\end{frame}

% TODO неверно, но лучше понять и описать правильный вариант этого подхода, чем fix (см. https://cs.stackexchange.com/questions/101152/let-rec-recursive-expression-static-typing-rule)
%\begin{frame}[fragile]
%\frametitle{Алгоритм $\J$: рекурсия по определению $e$}
%\begin{itemize}
%    \item Особый случай: рекурсивный \lstinline|let| ($x$ входит в $e_1$).
%    \pause
%    \begin{itemize}
%        \item Тогда перед выводом $e_1$ мы добавляем для $x$ переменную типа $\alpha$, а потом унифицируем $t_1$ с $\alpha$.
%        \pause
%        \item Это обобщается и на взаимную рекурсию.
%        \pause
%    \end{itemize}
%\end{itemize}
%\end{frame}

\begin{frame}[fragile]
\frametitle{Алгоритм $\J$: рекурсия по определению $e$}
\begin{itemize}
    \item Особый случай: рекурсивный \lstinline|let| ($x$ входит в $e_1$).
    \pause
    \item[]
    \begin{itemize}
        \item Используем функцию \lstinline|fix|.
        \item \lstinline|fix f| возвращает неподвижную точку \lstinline|f| (т.е. \lstinline|fix f x == x|). 
        \pause
        \item Соотвественно, её тип \pause \lstinline|fix :: forall a. (a -> a) -> a|.
        \pause
        \item \lstinline[mathescape]|let x = $e_1$ in $e_2$| превращается в \lstinline[mathescape]|let x = fix (\x -> $e_1$) in $e_2$|.
        \item Этот \lstinline|let| не рекурсивный, потому что
        \pause 
        \\ \lstinline|x| в \lstinline|let x = ...| и в \lstinline[mathescape]|\x -> $e_1$| "--- разные!
    \end{itemize}
\end{itemize}
\end{frame}

\begin{frame}[fragile]
\frametitle{Алгоритм $\J$: рекурсия по определению $e$}
\begin{itemize}
    \item \lstinline|case|-выражение
\begin{lstlisting}[mathescape]
case $e$ of
  $pat_1$ -> $body_1$
  $pat_2$ -> $body_2$
  ...
\end{lstlisting}
    \pause
    \begin{itemize}
        \item Выводим тип $t$ для $e$.
        \item Для $i$-той ветви:
        \begin{itemize}
            \pause
            \item Вводим переменные типов для каждой переменной в $pat_i$.
            \item Выводим тип для $pat_i$.
            \pause
            \item Унифицируем его с $t$.
            \pause
            \item Выводим тип $t_i$ для $body_i$.
        \end{itemize}
        \pause
        \item Унифицируем все $t_i$.
        \item Результат: \pause $t_1$ (или любой другой $t_i$).
    \end{itemize}
\end{itemize}
\end{frame}

%\begin{frame}[fragile]
%\frametitle{Алгоритм $\J$: рекурсия по определению $e$ (формально)}
%\begin{itemize}
%    \item 
%\begin{lstlisting}[mathescape]
%$\J(\Gamma; x)$ = do
%  $\forall \alpha_1\ldots\alpha_k t = \phi(\Gamma(x))$
%  return $t[\alpha_1 \mapsto \fresh,\ldots,\alpha_k \mapsto \fresh]$
%\end{lstlisting}
%    \item[] \pause
%    \item 
%\begin{lstlisting}[mathescape]
%$\J(\Gamma; e_1~e_2)$ = do
%  $t_1 = \J(\Gamma; e_1)$
%  $t_2 = \J(\Gamma; e_2)$
%  $\alpha = \fresh$
%  $\phi := \mathord{\mathit{unify}}(t_1 = t_2 \to \alpha) \circ \phi$
%  return $\phi(\alpha)$
%\end{lstlisting}
%\end{itemize}
%\end{frame}
%
%\begin{frame}[fragile]
%\frametitle{Алгоритм $\J$: рекурсия по определению $e$ (формально)}
%\begin{itemize}
%\item 
%\begin{lstlisting}[mathescape]
%$\J(\Gamma; \lambda x.e_1)$ = do
%  $\alpha = \fresh$
%  $t = \J(\Gamma, x:\alpha; e_1)$
%  return $\phi(\alpha \to t)$
%\end{lstlisting}
%    \item[] \pause
%\item 
%\begin{lstlisting}[mathescape]
%$\J(\Gamma; \mathtt{let}~x=e_1~\mathtt{in}~e_2)$ = do
%  $t_1 = \J(\Gamma; e_1)$
%  $\Gamma_1 = \phi(\Gamma)$
%  $fv = ftv(t_1) \setminus ftv(\Gamma_1)$
%  $t_2 = \forall fv.t_1$
%  return $\J(\Gamma_1, x:t_2; e_2)$
%\end{lstlisting}
%\end{itemize}
%\end{frame}

\begin{frame}[fragile]
\frametitle{Примеры}
\begin{itemize}
    \item Разберём несколько примеров
    \pause
    \item Без \lstinline|let|:
    \item[] \lstinline|f1 = (.) . (.)|
    \pause
    \item C \lstinline|let|-полиморфизмом:
    \item[] \lstinline|f2 x = let pair x = (x,x) in pair (pair x)|
    \pause
    \item Пример ошибки типа:
    \item[] \lstinline|f3 x = x x|
\end{itemize}
\end{frame}

\begin{frame}[fragile]
\frametitle{Дополнительное чтение}
\begin{itemize}
    \item Более известна альтернатива $\J$, избегающая \enquote{глобальной} $\phi$ "--- алгоритм $\mathcal{W}$. Вы можете легко найти его описания и реализации в Интернете.
    \item \href{http://dev.stephendiehl.com/fun/006_hindley_milner.html}{Глава Hindley-Milner Inference в Write You a Haskell}
    \item \href{http://citeseerx.ist.psu.edu/viewdoc/summary?doi=10.1.1.65.7733}{Algorithm W Step by Step}
    \item Реализация системы типов всего (!) стандарта Haskell 98 с выводом, основанная на $\J$, есть в \href{https://gist.github.com/chrisdone/0075a16b32bfd4f62b7b}{Typing Haskell in Haskell}.
    \item \href{https://compsciclub.ru/courses/types/2019-spring/classes/}{Курс лекций В.~Н.~Брагилевского "Вывод типов от Хиндли — Милнера до GHC 8.8"  (есть видео)}
    \item Реализации на Haskell обычно используют монады, но это не должно сильно осложнить понимание. 
\end{itemize}
\end{frame}

\end{document}