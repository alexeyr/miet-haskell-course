\documentclass[10pt]{beamer}
\usepackage[utf8]{inputenc}
\usepackage[T1,T2A]{fontenc}
\usepackage[russian]{babel}
\usepackage{color}
\usepackage{calc}
\usepackage{graphicx}
\usepackage{epstopdf}
\usepackage{hyperref}
\hypersetup{unicode,colorlinks}
\usetheme[progressbar=head,numbering=fraction,block=fill]{metropolis}
\usepackage{minted}
\usepackage{dejavu}
%\usepackage{adjustbox}  % Позволяет сузить куски кода (или текст) ровно настолько, чтобы уместиться в слайд
\usepackage{csquotes}
\usepackage{upquote}

\usemintedstyle{solarized-light}
\newminted[haskell]{haskell}{
    escapeinside=!!,
    mathescape=true,
    texcomments=true,
    beameroverlays=true,
    autogobble=true,
    fontsize=\small,
    breaklines=false  % Лучше сам поставлю переносы на удобных местах
}
\newminted[haskellsmall]{haskell}{
    escapeinside=!!,
    mathescape=true,
    texcomments=true,
    beameroverlays=true,
    autogobble=true,
    fontsize=\footnotesize,
    breaklines=false
}
\newminted[haskelltiny]{haskell}{
    escapeinside=!!,
    mathescape=true,
    texcomments=true,
    beameroverlays=true,
    autogobble=true,
    fontsize=\scriptsize,
    breaklines=false
}
\newmintinline[haskinline]{haskell}{
    escapeinside=!!,
    mathescape=true,
    beameroverlays=true,
    breaklines=true
}
\newminted[ghci]{text}{
    autogobble=true,
    fontsize=\small,
    breaklines=false
}
\newminted[ghcismall]{text}{
    autogobble=true,
    fontsize=\footnotesize,
    breaklines=false
}
\newminted[ghcitiny]{text}{
    autogobble=true,
    fontsize=\scriptsize,
    breaklines=false
}
\newmintinline[ghcinline]{text}{
    breaklines=true
}

\newcommand{\hackage}[1]{\href{https://hackage.haskell.org/package/#1}{#1}}

\vfuzz=20pt  % позволяет тексту дойти до номера слайда

\author{Алексей Романов}
\subtitle{Функциональное программирование на Haskell}
%\logo{}
\institute{МИЭТ}
\subject{Функциональное программирование на Haskell}
%\setbeamercovered{transparent}
%\setbeamertemplate{navigation symbols}{}


\title{Лекция 5: функции как значения}
\date{7 марта 2018}

\begin{document}
\begin{frame}[plain]
\maketitle
\end{frame}

\begin{frame}[fragile]
\frametitle{Функции как значения}
\begin{itemize}
    \item Как упоминалось в начале курса, одно из оснований ФП состоит в том, что функции могут использоваться как значения.
    \item В Haskell можно выразиться сильнее:\pause
    \item Функции это и есть просто значения, тип которых имеет форму \lstinline|ТипПараметра -> ТипРезультата| для каких-то \lstinline|ТипПараметра| и \lstinline|ТипРезультата|.\pause
    \item Мы уже видели примеры этого в равноправии функций и \only<4>{\textbf{других} }переменных.
\end{itemize}
\end{frame}

\begin{frame}[fragile]
\frametitle{Функции высших порядков}
\begin{itemize}
    \item В частности, функции могут принимать на вход функции.
    \item То есть тип параметра сам может быть функциональным типом. 
    \item Тривиальный пример:
\begin{lstlisting}
foo :: (Char -> Bool) -> Bool
foo f = f 'a'

Prelude Data.Char> foo isLetter !\pause!
True
\end{lstlisting} 
    \item Скобки вокруг типа параметра здесь необходимы.\pause
    \item Функции, параметры которых "--- функции, называются \emph{функциями высших порядков (ФВП)}.\pause
    \item Часто ими также считают функции, возвращающие функции, но в Haskell нет (скоро увидим почему).
\end{itemize}
\end{frame}

\begin{frame}[fragile]
\frametitle{Лямбда-выражения}
\begin{itemize}
    \item TODO
\end{itemize}
\end{frame}

\begin{frame}[fragile]
\frametitle{Функции применения и композиции функций}
\begin{itemize}
    \item TODO
\end{itemize}
\end{frame}

\begin{frame}[fragile]
\frametitle{Избавление от скобок}
\begin{itemize}
    \item TODO
\end{itemize}
\end{frame}

\begin{frame}[fragile]
\frametitle{Частичное применение}
\begin{itemize}
    \item TODO
\end{itemize}
\end{frame}

\begin{frame}[fragile]
\frametitle{$\eta$-редукция (сокращение аргументов)}
\begin{itemize}
    \item TODO
\end{itemize}
\end{frame}

\begin{frame}[fragile]
\frametitle{Сечения операторов}
\begin{itemize}
    \item TODO
\end{itemize}
\end{frame}

\begin{frame}[fragile]
\frametitle{Ограничение мономорфизмом}
\begin{itemize}
    \item TODO
\end{itemize}
\end{frame}

\begin{frame}[fragile]
\frametitle{Основные функции 2-го порядка над списками}
\begin{itemize}
    \item TODO
\end{itemize}
\end{frame}

\begin{frame}[fragile]
\frametitle{Бесточечный стиль}
\begin{itemize}
    \item TODO (нужно ли?)
\end{itemize}
\end{frame}

\end{document}