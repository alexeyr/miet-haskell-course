\documentclass[10pt]{beamer}
\usepackage[utf8]{inputenc}
\usepackage[T1,T2A]{fontenc}
\usepackage[russian]{babel}
\usepackage{color}
\usepackage{calc}
\usepackage{graphicx}
\usepackage{epstopdf}
\usepackage{hyperref}
\hypersetup{unicode,colorlinks}
\usetheme[progressbar=head,numbering=fraction,block=fill]{metropolis}
\usepackage{minted}
\usepackage{dejavu}
%\usepackage{adjustbox}  % Позволяет сузить куски кода (или текст) ровно настолько, чтобы уместиться в слайд
\usepackage{csquotes}
\usepackage{upquote}

\usemintedstyle{solarized-light}
\newminted[haskell]{haskell}{
    escapeinside=!!,
    mathescape=true,
    texcomments=true,
    beameroverlays=true,
    autogobble=true,
    fontsize=\small,
    breaklines=false  % Лучше сам поставлю переносы на удобных местах
}
\newminted[haskellsmall]{haskell}{
    escapeinside=!!,
    mathescape=true,
    texcomments=true,
    beameroverlays=true,
    autogobble=true,
    fontsize=\footnotesize,
    breaklines=false
}
\newminted[haskelltiny]{haskell}{
    escapeinside=!!,
    mathescape=true,
    texcomments=true,
    beameroverlays=true,
    autogobble=true,
    fontsize=\scriptsize,
    breaklines=false
}
\newmintinline[haskinline]{haskell}{
    escapeinside=!!,
    mathescape=true,
    beameroverlays=true,
    breaklines=true
}
\newminted[ghci]{text}{
    autogobble=true,
    fontsize=\small,
    breaklines=false
}
\newminted[ghcismall]{text}{
    autogobble=true,
    fontsize=\footnotesize,
    breaklines=false
}
\newminted[ghcitiny]{text}{
    autogobble=true,
    fontsize=\scriptsize,
    breaklines=false
}
\newmintinline[ghcinline]{text}{
    breaklines=true
}

\newcommand{\hackage}[1]{\href{https://hackage.haskell.org/package/#1}{#1}}

\vfuzz=20pt  % позволяет тексту дойти до номера слайда

\author{Алексей Романов}
\subtitle{Функциональное программирование на Haskell}
%\logo{}
\institute{МИЭТ}
\subject{Функциональное программирование на Haskell}
%\setbeamercovered{transparent}
%\setbeamertemplate{navigation symbols}{}


\title{Лекция 5: функции как значения}
\date{7 марта 2018}

\begin{document}
\begin{frame}[plain]
\maketitle
\end{frame}

\begin{frame}[fragile]
\frametitle{Функции как значения}
\begin{itemize}
    \item Как упоминалось в начале курса, одно из оснований ФП состоит в том, что функции могут использоваться как значения.
    \item В Haskell можно выразиться сильнее:\pause
    \item Функции это и есть просто значения, тип которых имеет форму \lstinline|ТипПараметра -> ТипРезультата| для каких-то \lstinline|ТипПараметра| и \lstinline|ТипРезультата|.\pause
    \item Мы уже видели примеры этого в равноправии функций и \only<4>{\textbf{других} }переменных.
\end{itemize}
\end{frame}

\begin{frame}[fragile]
\frametitle{Функции высших порядков}
\begin{itemize}
    \item В частности, функции могут принимать на вход функции.
    \item То есть тип параметра сам может быть функциональным типом. 
    \item Тривиальный пример:
\begin{lstlisting}
foo :: (Char -> Bool) -> Bool
foo f = f 'a'

Prelude Data.Char> foo isLetter !\pause!
True
\end{lstlisting} 
    \item Скобки вокруг типа параметра здесь необходимы.\pause
    \item Функции, параметры которых "--- функции, называются \emph{функциями высших порядков (ФВП)}.\pause
    \item Часто ими также считают функции, возвращающие функции, но в Haskell нет (скоро увидим почему).
\end{itemize}
\end{frame}

\begin{frame}[fragile]
\frametitle{Лямбда-выражения}
\begin{itemize}
    \item В \lstinline|foo| с прошлого слайда можем также передать свою новую функцию, определив её локально:
\begin{lstlisting}
foo isCyrillic where 
    isCyrillic c = 
        let lc = toLower c 
        in 'а' <= lc && lc <= 'я'
\end{lstlisting}\pause
    \item Но имя этой функции на самом деле не нужно.
    \item Как и вообще функциям, которые создаются только как аргументы для других (или как результаты).\pause
    \item Вместо этого зададим её через \emph{лямбда-выражение}
\begin{lstlisting}
foo (\c -> let lc = toLower c 
           in 'а' <= lc && lc <= 'я')
\end{lstlisting}\pause
    \pause
    \item[]
    \item Кто заметил ошибку в определении \lstinline|isCyrillic|?
\end{itemize}
\end{frame}

\begin{frame}[fragile]
\frametitle{Лямбда-выражения}
\begin{itemize}
    \item Вообще, два определения
\begin{lstlisting}
функция образец = результат

функция = \образец -> результат
\end{lstlisting}
    эквивалентны.\pause
    \item Одно исключение: \href{https://wiki.haskell.org/Monomorphism_restriction}{для второго может быть выведен менее общий тип}.\pause
    \item Лямбда-выражение для функции с несколькими параметрами пишется
\begin{lstlisting}
\образец1 ... образецN -> результат
\end{lstlisting}\pause
    \item Например, 
\begin{lstlisting}
Data.List.sortBy (\x y -> compare y x) list
\end{lstlisting}
    \item Что делает это выражение?\pause
    \item Сортирует список по убыванию.
\end{itemize}
\end{frame}

\begin{frame}[fragile]
\frametitle{Лямбда-выражения и \lstinline|case|}
\begin{itemize}
    \item Если в обычном определении функции несколько уравнений, например
\begin{lstlisting}
not True  = False
not False = True
\end{lstlisting}
    то в лямбда-выражении придётся использовать \lstinline|case|:\pause
\begin{lstlisting}
not = \x -> case x of
    True  -> False
    False -> True
\end{lstlisting}\pause
    или с расширением \lstinline|LambdaCase|
\begin{lstlisting}
not = \case
    True  -> False
    False -> True
\end{lstlisting}
\end{itemize}
\end{frame}

\begin{frame}[fragile]
\frametitle{Несколько стандартных ФВП}
\begin{itemize}
    \item В \lstinline|Prelude| есть три функции второго порядка, которые очень часто встречаются в коде Haskell:
\begin{lstlisting}
($) :: (a -> b) -> a -> b
f $ x =!\pause! f x!\pause!

(.) :: (b -> c) -> (a -> b) -> (a -> c)
g . f =!\pause! \x ->!\pause! g (f x)

flip :: (a -> b -> c) -> (b -> a -> c)
flip f = !\pause! \y x ->!\pause! f x y
\end{lstlisting}
    \item И ещё две в \lstinline|Data.Function|:
\begin{lstlisting}
(&) = flip (.)
(&) ::!\pause! (a -> b) -> (b -> c) -> (a -> c)

on :: (b -> b -> c) -> (a -> b) -> 
    (a -> a -> c)
\end{lstlisting}
\end{itemize}
\end{frame}

\begin{frame}[fragile]
\frametitle{Избавление от скобок с помощью \lstinline|$| и \lstinline|.|}
\begin{itemize}
    \item Казалось бы, в чём смысл \lstinline|$|: зачем писать \lstinline|f $ x| вместо \lstinline|f x|?\pause
    \item Этот оператор имеет минимальный возможный приоритет, так что \lstinline|f (x + y)| можно записать как \lstinline|f $ x + y|.\pause
    \item И он правоассоциативен, так что \lstinline|f (g (h x))| можно записать как \lstinline|f $ g $ h x|.\pause
    \item Но более принято \lstinline|f . g . h $ x|.\pause
    \item \lstinline|.| тоже правоассоциативен, но уже с максимальным приоритетом.\pause
    \item[]
    \item Пока, наверное, проще читать и писать код со скобками, но стандартный стиль Haskell предпочитает их избегать.
    \item Только не перестарайтесь!
\end{itemize}
\end{frame}

\begin{frame}[fragile]
\frametitle{Правда о функциях многих переменных}
\begin{itemize}
    \item Настала пора раскрыть тайну функций многих переменных в Haskell:\pause
    \item Их не существует.\pause
    \item \lstinline|->| "--- правоассоциативный оператор, так что
    \item[] \lstinline|Тип1 -> Тип2 -> Тип3| это на самом деле \lstinline|Тип1 -> (Тип2 -> Тип3)|: функция, возвращающая функцию.\pause
    \item \lstinline|\x y -> результат| это сокращение для \lstinline|\x -> \y -> результат|, а со скобками \lstinline|\x -> (\y -> результат)|.\pause
    \item Применение функций (не \lstinline|$|, а пробел), наоборот, левоассоциативно. То есть \lstinline|f x y| читается как \lstinline|(f x) y|.
\end{itemize}
\end{frame}

\begin{frame}[fragile]
\frametitle{Каррирование}
\begin{itemize}
    \item Обычно в математике для сведения функций нескольких аргументов к функциям одного аргумента используется декартово произведение: $A \times B \to C$, а не $A \to (B \to C)$.\pause
    \item В Haskell тоже можно было бы писать
\begin{lstlisting}
foo :: (Int, Int) -> Int
foo (x, y) = x + y
\end{lstlisting}
для определения функций и \lstinline|foo (1, 2)| для вызова.\pause
    \item Но Haskell здесь следует традиции $\lambda$-исчисления.\pause
    \item Эти подходы эквивалентны, так как множества $A \times B \to C$ и $A \to (B \to C)$ всегда изоморфны \pause ($C^{A \cdot B} = C^{B^A}$).
    \item В Haskell этот изоморфизм реализуют функции\pause
\begin{lstlisting}
curry :: ((a, b) -> c) -> a -> b -> c
uncurry :: (a -> b -> c) -> (a, b) -> c
\end{lstlisting}
\end{itemize}
\end{frame}

\begin{frame}[fragile]
\frametitle{Частичное применение}
\begin{itemize}
    \item Преимущество каррированных функций в том, что естественным образом появляется частичное применение.
    \item То есть мы можем применить функцию \enquote{двух аргументов} только к первому и останется функция одного аргумента:
\begin{lstlisting}
Prelude Data.List> :t sortBy
sortBy :: (a -> a -> Ordering) -> [a] -> [a]
Prelude Data.List> :t sortBy (flip compare) !\pause!
sortBy (flip compare) :: Ord a => [a] -> [a]
\end{lstlisting}\pause
    \item Заметьте, что здесь частичное применение в двух местах: \pause\lstinline|flip| можно теперь рассматривать как функцию трёх аргументов!
\end{itemize}
\end{frame}

\begin{frame}[fragile]
\frametitle{Сечения операторов}
\begin{itemize}
    \item Для применения бинарного оператора только к первому аргументу можно использовать его префиксную форму:
\begin{lstlisting}
Prelude> :t (+) 1
(+) 1 :: Num a => a -> a
\end{lstlisting}\pause
    \item А ко второму? Можно использовать лямбду
\begin{lstlisting}
\x -> x / 2
\end{lstlisting}
или \lstinline|flip|
\begin{lstlisting}
flip (/) 2
\end{lstlisting}\pause
    \item Но есть специальный синтаксис \lstinline|(арг оп)| и \lstinline|(оп арг)|:
\begin{lstlisting}
Prelude> (1 `div`) 2 !\pause!
0
Prelude> (/ 2) 2 !\pause!
1.0
\end{lstlisting}
\end{itemize}
\end{frame}

\begin{frame}[fragile]
\frametitle{Сечения кортежей}
\begin{itemize}
    \item Расширение \lstinline|TupleSections| позволяет частично применять конструкторы кортежей:
\begin{lstlisting}
Prelude> :set -XTupleSections
Prelude> (, "I", , "Love", True) 1 False
(1,"I",False,"Love",True)
Prelude> :t (, "I", , "Love", True) !\pause!
(, "I", , "Love", True)
  :: t1 -> t2 -> (t1, [Char], t2, [Char], Bool)
\end{lstlisting}
\end{itemize}
\end{frame}

\begin{frame}[fragile]
\frametitle{$\eta$-эквивалентность (сокращение аргументов)}
\begin{itemize}
    \item Рассмотрим два определения
\begin{lstlisting}
foo' x = foo x
-- или foo' = \x -> foo x 
\end{lstlisting}
    \item \lstinline|foo' y == foo y|, какое бы \lstinline|y| (и \lstinline|foo|) мы не взяли.\pause
    \item Поэтому мы можем упростить определение:
\begin{lstlisting}
foo' = foo
\end{lstlisting}\pause
    \item Это также относится к случаям, когда совпадают часть аргументов в конце:
\begin{lstlisting}
foo' x y z w = foo (y + x) z w
\end{lstlisting}
эквивалентно
\begin{lstlisting}
foo' x y = foo (y + x)
\end{lstlisting}
\end{itemize}
\end{frame}

\begin{frame}[fragile]
\frametitle{$\eta$-эквивалентность (сокращение аргументов)}
\begin{itemize}
    \item И когда само определение такой формы не имеет, но может быть к ней преобразовано
\begin{lstlisting}
root4 x = sqrt (sqrt x)
\end{lstlisting}
можно переписать как
\begin{lstlisting}
root4 x = (sqrt . sqrt) x

root4 = sqrt . sqrt
\end{lstlisting}
    \item Ещё пример:
\begin{lstlisting}
($) :: (a -> b) -> a -> b
f $ x = f x

($) f x = f x !\pause!
($) f = f !\pause!
($) f = id f !\pause!
($) = id
\end{lstlisting}
\end{itemize}
\end{frame}

\begin{frame}[fragile]
\frametitle{Бесточечный стиль}
\begin{itemize}
    \item Как видим, с помощью композиции и других операций можно дать определение некоторых функций без использования переменных.
    \item Это называется \emph{бесточечным стилем} (переменные рассматриваются как точки в пространстве значений).\pause
    \item Оказывается, что очень многие выражения в Haskell имеют бесточечный эквивалент.
    \item На \href{https://wiki.haskell.org/Pointfree}{Haskell Wiki} можно найти инструменты, позволяющие переводить между стилями.
    \item Опять же, не перестарайтесь:
\begin{lstlisting}
> pl \x y -> compare (f x) (f y)
((. f) . compare .)
\end{lstlisting}\pause
    \item В языках семейств Forth и APL бесточечный стиль является основным.
\end{itemize}
\end{frame}

\begin{frame}[fragile]
\frametitle{Программирование, направляемое типами}
\begin{itemize}
    \item TODO
\end{itemize}
\end{frame}

\begin{frame}[fragile]
\frametitle{Типизированные дыры в коде}
\begin{itemize}
    \item TODO
\end{itemize}
\end{frame}

%\begin{frame}[fragile]
%\frametitle{Основные функции 2-го порядка над списками}
%\begin{itemize}
%    \item TODO
%\end{itemize}
%\end{frame}

\end{document}