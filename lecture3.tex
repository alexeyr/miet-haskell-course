\documentclass[10pt]{beamer}
\usepackage[utf8]{inputenc}
\usepackage[T1,T2A]{fontenc}
\usepackage[russian]{babel}
\usepackage{color}
\usepackage{calc}
\usepackage{graphicx}
\usepackage{epstopdf}
\usepackage{hyperref}
\hypersetup{unicode,colorlinks}
\usetheme[progressbar=head,numbering=fraction,block=fill]{metropolis}
\usepackage{minted}
\usepackage{dejavu}
%\usepackage{adjustbox}  % Позволяет сузить куски кода (или текст) ровно настолько, чтобы уместиться в слайд
\usepackage{csquotes}
\usepackage{upquote}

\usemintedstyle{solarized-light}
\newminted[haskell]{haskell}{
    escapeinside=!!,
    mathescape=true,
    texcomments=true,
    beameroverlays=true,
    autogobble=true,
    fontsize=\small,
    breaklines=false  % Лучше сам поставлю переносы на удобных местах
}
\newminted[haskellsmall]{haskell}{
    escapeinside=!!,
    mathescape=true,
    texcomments=true,
    beameroverlays=true,
    autogobble=true,
    fontsize=\footnotesize,
    breaklines=false
}
\newminted[haskelltiny]{haskell}{
    escapeinside=!!,
    mathescape=true,
    texcomments=true,
    beameroverlays=true,
    autogobble=true,
    fontsize=\scriptsize,
    breaklines=false
}
\newmintinline[haskinline]{haskell}{
    escapeinside=!!,
    mathescape=true,
    beameroverlays=true,
    breaklines=true
}
\newminted[ghci]{text}{
    autogobble=true,
    fontsize=\small,
    breaklines=false
}
\newminted[ghcismall]{text}{
    autogobble=true,
    fontsize=\footnotesize,
    breaklines=false
}
\newminted[ghcitiny]{text}{
    autogobble=true,
    fontsize=\scriptsize,
    breaklines=false
}
\newmintinline[ghcinline]{text}{
    breaklines=true
}

\newcommand{\hackage}[1]{\href{https://hackage.haskell.org/package/#1}{#1}}

\vfuzz=20pt  % позволяет тексту дойти до номера слайда

\author{Алексей Романов}
\subtitle{Функциональное программирование на Haskell}
%\logo{}
\institute{МИЭТ}
\subject{Функциональное программирование на Haskell}
%\setbeamercovered{transparent}
%\setbeamertemplate{navigation symbols}{}


\title{Лекция 3: типы данных}

\begin{document}
\begin{frame}[plain]
  \maketitle
\end{frame}

\begin{frame}[fragile]
  \frametitle{Типы-\enquote{перечисления}}
  \begin{itemize}
    \item Тип \haskinline|Bool| не встроен в язык, а определён в стандартной библиотеке:
          \begin{haskell}
            data Bool = False | True deriving (...)
          \end{haskell}
    \item Ключевое слово \haskinline|data| начинает определение.
    \item \haskinline|Bool| "--- название типа.
    \item \haskinline|False| и \haskinline|True| называются конструкторами данных.
    \item Читаем как \enquote{У типа \haskinline|Bool| есть ровно два значения: \haskinline|False| и \haskinline|True|.}
          \only<1> { \item О \haskinline|deriving| позже. }
          \pause
    \item Так определяется любой тип с фиксированным перечнем значений (как \haskinline|enum| в других языках):
          \begin{haskell}
            data Weekday = Monday | Tuesday | ...
            data FileMode = Read | Write | Append | ...
          \end{haskell}
    \item Какие образцы у этих типов (кроме переменных и \haskinline|_|)? \pause
    \item Каждое значение (конструктор) "--- образец.
  \end{itemize}
\end{frame}

\begin{frame}[fragile]
  \frametitle{Типы-\enquote{структуры}}
  \begin{itemize}
    \item \haskinline|data| также используется для аналогов структур: типов с несколькими полями. Например,
          \begin{haskell}
data Point = Point Double Double
\end{haskell}
    \item Первое \haskinline|Point| "--- название типа.
    \item Второе "--- конструктора.
    \item Когда конструктор один, названия обычно совпадают. Всегда по контексту можно определить, что из них имеется в виду.
    \item Образцы этого типа \pause это\\
          \begin{haskell}
            Point образец_Double образец_Double
          \end{haskell}
    \item Зададим значение типа \haskinline|Point| и функцию на них:
          \begin{haskell}
            ghci> origin = Point 0 0

            ghci> xCoord (Point x _) = x
            ghci> xCoord origin
            0.0
          \end{haskell}
  \end{itemize}
\end{frame}

\begin{frame}[fragile]
  \frametitle{Типы-\enquote{структуры}}
  \begin{itemize}
    \item Конструктор является функцией
          \begin{haskell}
            ghci> :t Point !\pause!
            Point :: Double -> Double -> Point
          \end{haskell}
    \item Мы можем дать полям конструктора имена. Это:\hypertarget{rec1}{}
          \begin{itemize}
            \item Документирует их смысл.
            \item Определяет функции, возвращающие их значения.
          \end{itemize}
          \begin{haskell}
            ghci> data Point = Point { xCoord :: Double, yCoord :: Double } deriving Show
            ghci> :t yCoord
            yCoord :: Point -> Double
            ghci> yCoord (Point 0 (-1))
            -1.0
          \end{haskell}
    \item Особенно полезно, когда полей много, и только по типу не понять, какое где.
    \item Если успеем, вернёмся к записям в конце лекции. \hyperlink{rec2}{\beamergotobutton{Синтаксис записей}}
  \end{itemize}
\end{frame}

\begin{frame}[fragile]
  \frametitle{Алгебраические типы: общий случай}
  \begin{itemize}
    \item Может быть несколько конструкторов с полями:
          \begin{haskell}
            data IpAddress = IPv4 String | IPv6 String
          \end{haskell}
    \item Создание значения:
          \begin{haskell}
ghci> googleDns = IPv4 "8.8.8.8"
\end{haskell}
    \item Образцы: \haskinline|IPv4 образец_String| и \haskinline|IPv6 образец_String|.
    \item Функции часто определяются по уравнению на каждый конструктор:
          \begin{haskell}
            asString (IPv4 ip4) = ip4
            asString (IPv6 ip6) = ip6
          \end{haskell}
          \pause
    \item Но не обязательно:
          \begin{haskell}
            isLocalhost (IPv4 "127.0.0.1") = True
            isLocalhost (IPv6 "0:0:0:0:0:0:0:1") = True
            isLocalhost _ = False
          \end{haskell}
          \pause
    \item Ещё пример: \haskinline!data ImageFormat = JPG | PNG | IF String!
  \end{itemize}
\end{frame}

\begin{frame}[fragile]
  \frametitle{Типы с параметрами (полиморфные)}
  \begin{itemize}
    \item В определении типа могут быть параметры.
    \item Пример:
          \begin{haskell}
            data Maybe a = Nothing | Just a
          \end{haskell}
    \item \haskinline|Maybe| "--- конструктор типа.
    \item Вместо \haskinline|a| можно подставить любой тип, и получить снова тип: \haskinline|Maybe Int|, \haskinline|Maybe Bool|.
    \item Примеры их значений: \haskinline|Just 1|, \haskinline|Nothing|.
    \item В Haskell нет отношения \enquote{Надтип "--- подтип}, вместо него \enquote{Более общий "--- частный случай}.
    \item \haskinline|Maybe a| описывает необязательное значение типа \haskinline|a|.
          \pause
    \item Ещё часто встречающийся тип:
          \begin{haskell}
            data Either a b = Left a | Right b
          \end{haskell}
    \item Могут быть сколь угодно сложные комбинации.
    \item Например, \haskinline|Either (Either Int Char) (Maybe Bool)| со значением \haskinline|Right Nothing|.
  \end{itemize}
\end{frame}

\begin{frame}[fragile]
  \frametitle{\haskinline|Maybe| и \haskinline|null|}
  \begin{itemize}
    \item \haskinline|Nothing| похоже на \haskinline|null| в C-подобных языках, но явно прописано в типах.
    \item Не нужно для каждой функции документировать, как работает с \haskinline|null|, достаточно посмотрить на наличие \haskinline|Maybe| в сигнатуре.
    \item И поэтому нет возможного несовпадения реализации с документацией.
    \item Нет типов, у которых \haskinline|null| нет и с отсутствием значения приходится что-то придумывать.
    \item Есть \haskinline|Maybe (Maybe a)| (бывает полезно).
          \pause
    \item \enquote{\itshape \haskinline|null| "--- моя ошибка ценой в миллиард долларов... Я планировал сделать любое использование ссылок [в Algol] абсолютно безопасным. Но не удержался от соблазна добавить \haskinline|null| просто потому, что его было так легко реализовать.} (\href{https://ru.wikipedia.org/wiki/Хоар,_Чарльз_Энтони_Ричард}{Тони Хоар}, \href{https://www.infoq.com/presentations/Null-References-The-Billion-Dollar-Mistake-Tony-Hoare}{QCon 2009})
  \end{itemize}
\end{frame}

\begin{frame}[fragile]
  \frametitle{Кортежи}
  \begin{itemize}
    \item Мы могли бы определить типы
          \begin{haskell}
            data Pair a b = Pair a b
            data Triple a b c = Triple a b c
            ...
          \end{haskell}
          (заметьте разницу смыслов \haskinline|Pair|, \haskinline|a| и \haskinline|b| в левой и правой частях!)
    \item Это универсальные произведения.
          \pause
    \item Но определять их не нужно, так как такие типы уже есть, со специальным синтаксисом:\\
          \haskinline|(a, b)| (или \haskinline|(,) a b|),\\ \haskinline|(a, b, c)| (или \haskinline|(,,) a b c|) и т.д.
    \item Синтаксис значений и образцов для них аналогичный:
          \haskinline|(True, Just 'a') :: |\pause\haskinline|(Bool, Maybe Char)|.
    \item Максимальный размер 62.
  \end{itemize}
\end{frame}

\begin{frame}[fragile]
  \frametitle{Рекурсивные типы и списки}
  \begin{itemize}
    \item Тип может быть рекурсивным, то есть в правой части используется тот тип, который мы определяем.
    \item Самый важный пример такого типа: списки.
          Мы могли бы их определить как
          \begin{haskell}
            data List a = Nil | Cons a (List a)
          \end{haskell}
          Вместо этого они имеют специальный синтаксис, как и кортежи, как будто их определение
          \begin{haskell}
            data [a] = [] | a : [a]
          \end{haskell}
    \item Любой \haskinline|[a]| или пуст, или имеет \emph{голову} типа \haskinline|a| и \emph{хвост} типа \haskinline|[a]|.
    \item \haskinline|:| правоассоциативно: \haskinline|1 : 2 : 4 : []| читается \haskinline|1 : (2 : (4 : []))|.
    \item Сокращение (относится и к значениям и к образцам):\\
          \haskinline|[x, y, z]| означает \haskinline|x : y : z : []|, а \haskinline|[x]| "--- \haskinline|x : []|.
  \end{itemize}
\end{frame}

\begin{frame}[fragile]
  \frametitle{Синонимы типов}
  \begin{itemize}
    \item На этом с \haskinline|data| разобрались (пока что).
          \pause
    \item Есть ещё два вида определения типов.
    \item \haskinline|type| объявляет синоним типа "--- другое имя для существующего типа.
    \item В стандартной библиотеке:
          \begin{haskell}
            type String = [Char]
            type FilePath = String
          \end{haskell}
    \item Оба определения теперь считаются плохими! Для \haskinline|String| подробности в будущих лекциях.
          \pause
    \item Для \haskinline|FilePath| дело в том, что осмысленные операции над путями не те, что над строками:
          \begin{haskell}
            ghci> isPrefixOf 
                ("C:\\Program Files" :: FilePath) 
                ("C:\\Program Files (x86)" :: FilePath)
            True
          \end{haskell}
  \end{itemize}
\end{frame}

\begin{frame}[fragile]
  \frametitle{\haskinline|newtype|}
  \begin{itemize}
    \item \haskinline|newtype| как раз позволяет определить тип с тем же представлением, что существующий, но с другими операциями. Т.е. разумнее было бы
          \begin{haskell}
            newtype FilePath = FilePath String
          \end{haskell}
    \item Или \haskinline|FilePath [String]| со списком директорий.
    \item Ещё можно рассмотреть
          \begin{haskell}
            newtype EMail = EMail String
          \end{haskell}
    \item У \haskinline|type| конструкторов значений нет, у \haskinline|newtype| "--- всегда один с одним полем.
    \item Между \haskinline|newtype| и \haskinline|data| с одним конструктором с одним полем есть разница из-за ленивости.
    \item Пока из них предпочитаем первое.
  \end{itemize}
\end{frame}

\begin{frame}[fragile]
  \frametitle{Полиморфные функции и значения}
  \begin{itemize}
    \item Какой тип функции
          \begin{haskell}
            ghci> id x = x
            ghci> :t id !\pause!
            id :: a -> a
          \end{haskell}
    \item Более явно: \haskinline|id :: forall a. a -> a|.
    \item Аналогично
          \begin{haskell}
            ghci> foo (_, x, y) = (y, x, x)
            ghci> :t foo !\pause!
            id :: (a1, c, a2) -> (a2, c, c) 
            -- или (a, b, c) -> (c, b, b)
          \end{haskell}
    \item \haskinline|Just :: a -> Maybe a|
    \item И \haskinline|[] :: [a]| "--- полиморфные значения не обязательно функции.
    \item Параметр типа можно передать явно:
          \begin{haskell}
            ghci> :t id @Int
            id @Int :: Int -> Int
          \end{haskell}
  \end{itemize}
\end{frame}

\begin{frame}[fragile]
  \frametitle{Область видимости переменных типа}
  \begin{itemize}
    \item По умолчанию область видимости переменной типа "--- сигнатура, где она появилась. Например, в
          \begin{haskell}
            g :: [a] -> [a]
            g (x:xs) = xs ++ [x :: a]
          \end{haskell}
          \pause
          две переменные \haskinline|a| "--- разные и читаются как
          \begin{haskell}
            g :: forall a. [a] -> [a]
            g (x:xs) = xs ++ [x :: forall a. a]
          \end{haskell}
          \pause
          (так что функция не скомпилируется).
          \pause
    \item Но можно написать так:
          \begin{haskell}
            g :: forall a. [a] -> [a]
            g (x:xs) = xs ++ [x :: a]
          \end{haskell}
    \item Точные правила можно найти в \href{https://ghc.gitlab.haskell.org/ghc/doc/users_guide/exts/scoped_type_variables.html}{Lexically scoped type variables}.
  \end{itemize}
\end{frame}

\begin{frame}[fragile]
  \frametitle{Структурно-рекурсивные функции}
  \begin{itemize}
    \item Функции над рекурсивными типами часто тоже рекурсивны (или вызывают рекурсивные).
          \begin{haskell}
            length :: [a] -> Int
            length [] = 0
            length (_ : xs) = length xs + 1
          \end{haskell}
    \item Натуральные числа "--- \enquote{почётный} рекурсивный тип.
          \pause
    \item Натуральное число это либо $0$, либо $n + 1$, где $n$ "--- натуральное число.
    \item Соответственно, функции над ними тоже определяются рекурсивно:
          \begin{haskell}
            replicate _ 0 = []
            replicate x n = x : replicate x (n - 1)
          \end{haskell}
    \item Раньше \haskinline|n + 1| (и вообще \haskinline|+ константа|) было образцом, в Haskell 2010 их исключили.
  \end{itemize}
\end{frame}

\begin{frame}[fragile]
  \frametitle{Алгебраические типы}
  \begin{itemize}
    \item Несколько полей у конструктора соответствуют операции декартова произведения в теории множеств.
          \onslide<3->{\begin{itemize}
              \item В логике конъюнкции, а в алгебре произведению.
            \end{itemize}}
    \item А несколько конструкторов\onslide<1>{?}\onslide<2->{операции дизъюнктного объединения $A_1 \sqcup A_2 \sqcup \ldots = \{1\} \times A_1 \cup \{2\} \times A_2 \cup \ldots$}
          \onslide<3->{
            \begin{itemize}
              \item В логике дизъюнкции, а в алгебре сумме.
            \end{itemize}
          }
          \onslide<2->{
    \item Сравните помеченное объединение с непомеченным \haskinline|union| в C/C++.
    \item И с \haskinline|std::variant|.
          }
          \onslide<3->{\item Тип функций "--- множество функций в теории множеств, импликация в логике и возведение в степень в алгебре.}
          \onslide<4->{\item Обычные законы алгебры выполняются для типов с точностью до изоморфизма, как и для множеств.}
          % TODO переделать в таблицу
  \end{itemize}
\end{frame}

\begin{frame}[fragile]
  \frametitle{Типы и модули}
  \begin{itemize}
    \item Типы можно импортировать и экспортировать.
    \item \haskinline|НазваниеТипа(..)| в списке экспорта указывает, что конструкторы тоже включены в список.
    \item Просто \haskinline|НазваниеТипа| экспортирует только тип без конструкторов.
    \item Можно также перечислить явно, какие именно конструкторы экспортируются.
    \item Для импорта всё аналогично.
  \end{itemize}
\end{frame}

\begin{frame}[fragile]
  \frametitle{Записи}\hypertarget{rec2}{}
  \hyperlink{rec1}{\beamerreturnbutton{К определению записей}}
  \begin{itemize}
    \item
          \begin{haskellsmall}
            ghci> aPoint = Point 0 (-1)
            ghci> aPoint { xCoord = 1 } !\pause!
            Point {xCoord = 1.0, yCoord = -1.0}
            ghci> aPoint !\pause!
            Point {xCoord = 0.0, yCoord = -1.0} !\pause!
            ghci> Point {xCoord = x'} = it
            ghci> x' !\pause!
            0.0 !\pause!
            ghci> :set -XOverloadedRecordDot -XNamedFieldPuns -XRecordWildCards
            ghci> aPoint.xCoord
            0.0 !\pause!
            ghci> :set 
            ghci> f (Point { xCoord, yCoord }) = xCoord + yCoord
            ghci> g (Point { .. }) = xCoord + yCoord
          \end{haskellsmall}
    \item Два последних образца "--- сокращения\\ \haskinline|Point { xCoord = xCoord, yCoord = yCoord }|.
  \end{itemize}
\end{frame}

\begin{frame}[fragile]
  \frametitle{Записи: минусы}\hypertarget{rec2}{}
  \hyperlink{rec1}{\beamerreturnbutton{К определению записей}}
  \begin{itemize}
    \item
          \begin{haskell}
ghci> Point { xCoord = 0 }
\end{haskell}
          \pause
          без \haskinline|yCoord| компилируется только с предупреждением (можно сделать ошибкой с помощью опций компилятора).
          \pause
    \item Записи можно применять для типов с несколькими конструкторами, но это создаёт частичные (не всюду определённые) функции.
          \pause
    \item Одинаковые названия полей у двух записей ведут к проблемам.
          \pause
    \item Получается две функции с одинаковыми названиями.
    \item \haskinline|DisambiguateRecordFields| и \haskinline|DuplicateRecordFields| улучшают ситуацию.
          \pause
    \item Есть ещё решения на уровне библиотек.
  \end{itemize}
\end{frame}

\end{document}
