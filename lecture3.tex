\documentclass[10pt]{beamer}
\usepackage[utf8]{inputenc}
\usepackage[T1,T2A]{fontenc}
\usepackage[russian]{babel}
\usepackage{color}
\usepackage{calc}
\usepackage{graphicx}
\usepackage{epstopdf}
\usepackage{hyperref}
\hypersetup{unicode,colorlinks}
\usetheme[progressbar=head,numbering=fraction,block=fill]{metropolis}
\usepackage{minted}
\usepackage{dejavu}
%\usepackage{adjustbox}  % Позволяет сузить куски кода (или текст) ровно настолько, чтобы уместиться в слайд
\usepackage{csquotes}
\usepackage{upquote}

\usemintedstyle{solarized-light}
\newminted[haskell]{haskell}{
    escapeinside=!!,
    mathescape=true,
    texcomments=true,
    beameroverlays=true,
    autogobble=true,
    fontsize=\small,
    breaklines=false  % Лучше сам поставлю переносы на удобных местах
}
\newminted[haskellsmall]{haskell}{
    escapeinside=!!,
    mathescape=true,
    texcomments=true,
    beameroverlays=true,
    autogobble=true,
    fontsize=\footnotesize,
    breaklines=false
}
\newminted[haskelltiny]{haskell}{
    escapeinside=!!,
    mathescape=true,
    texcomments=true,
    beameroverlays=true,
    autogobble=true,
    fontsize=\scriptsize,
    breaklines=false
}
\newmintinline[haskinline]{haskell}{
    escapeinside=!!,
    mathescape=true,
    beameroverlays=true,
    breaklines=true
}
\newminted[ghci]{text}{
    autogobble=true,
    fontsize=\small,
    breaklines=false
}
\newminted[ghcismall]{text}{
    autogobble=true,
    fontsize=\footnotesize,
    breaklines=false
}
\newminted[ghcitiny]{text}{
    autogobble=true,
    fontsize=\scriptsize,
    breaklines=false
}
\newmintinline[ghcinline]{text}{
    breaklines=true
}

\newcommand{\hackage}[1]{\href{https://hackage.haskell.org/package/#1}{#1}}

\vfuzz=20pt  % позволяет тексту дойти до номера слайда

\author{Алексей Романов}
\subtitle{Функциональное программирование на Haskell}
%\logo{}
\institute{МИЭТ}
\subject{Функциональное программирование на Haskell}
%\setbeamercovered{transparent}
%\setbeamertemplate{navigation symbols}{}


\title{Лекция 3: типы данных}
\date{21 февраля 2018}

\begin{document}
\begin{frame}[plain]
\maketitle
\end{frame}

\begin{frame}[fragile]
\frametitle{Типы-\enquote{перечисления}}
\begin{itemize}
    \item Тип \lstinline|Bool| не встроен в язык, а определён в стандартной библиотеке:
\begin{lstlisting}
data Bool = False | True deriving (...)
\end{lstlisting}
    \item Ключевое слово \lstinline|data| начинает определение.
    \item \lstinline|Bool| "--- название типа.
    \item \lstinline|False| и \lstinline|True| называются конструкторами данных.
    \item Читаем как \enquote{У типа \lstinline|Bool| есть ровно два значения: \lstinline|False| и \lstinline|True|.} 
    \only<1> { \item О \lstinline|deriving| позже. }
    \pause
    \item Так определяется любой тип с фиксированным перечнем значений (как \lstinline|enum| в других языках):
\begin{lstlisting}
data Weekday = Monday | Tuesday | ...
data FileMode = Read | Write | Append | ...
\end{lstlisting}
    \item Какие образцы у этих типов (кроме переменных и \lstinline|_|)? \pause
    \item Каждое значение (конструктор) "--- образец.
\end{itemize}
\end{frame}

\begin{frame}[fragile]
\frametitle{Типы-\enquote{структуры}}
\begin{itemize}
    \item \lstinline|data| также используется для аналогов структур: типов с несколькими полями. Например,
\begin{lstlisting}[basicstyle=\ttfamily\footnotesize]
data Point = Point Double Double
\end{lstlisting}
    \item Первое \lstinline|Point| "--- название типа.
    \item Второе "--- конструктора.
    \item Когда конструктор один, названия обычно совпадают. Всегда по контексту можно определить, что из них имеется в виду.
    \item Образцы этого типа \pause это\\
\begin{lstlisting}[basicstyle=\ttfamily\footnotesize]
Point образец_Double образец_Double
\end{lstlisting}
    \item Зададим значение типа \lstinline|Point| и функцию на них:
\begin{lstlisting}[basicstyle=\ttfamily\footnotesize]
Prelude> let origin = Point 0 0

Prelude> let xCoord (Point x _) = x
Prelude> xCoord origin
0.0
\end{lstlisting}
\end{itemize}
\end{frame}

\begin{frame}[fragile]
\frametitle{Типы-\enquote{структуры}}
\begin{itemize}
    \item Конструктор является функцией
\begin{lstlisting}
Prelude> :t Point !\pause!
Point :: Double -> Double -> Point
\end{lstlisting}
    \item Мы можем дать полям конструктора имена. Это:\hypertarget{rec1}{}
    \begin{itemize}
        \item Документирует их смысл.
        \item Определяет функции, возвращающие их значения.
    \end{itemize}
\begin{lstlisting}
Prelude> data Point = Point { xCoord :: Double, yCoord :: Double } deriving Show
Prelude> :t yCoord
yCoord2 :: Point -> Double
Prelude> yCoord (Point 0 (-1))
-1.0
\end{lstlisting}
\item Особенно полезно, когда полей много, и только по типу не понять, какое где.
\item Если успеем, вернёмся к записям в конце лекции. \hyperlink{rec2}{\beamergotobutton{Синтаксис записей}}
\end{itemize}
\end{frame}

\begin{frame}[fragile]
\frametitle{Алгебраические типы: общий случай}
\begin{itemize}
    \item Может быть несколько конструкторов с полями:
\begin{lstlisting}[basicstyle=\ttfamily\small]
data IpAddress = IPv4 String | IPv6 String
\end{lstlisting}
\item Читаем \enquote{У типа \lstinline|IpAddress| есть два \emph{вида} значений: \lstinline|IPv4| и \lstinline|IPv6|.} Создание значения:
\begin{lstlisting}[basicstyle=\ttfamily\small]
Prelude> let googleDns = IPv4 "8.8.8.8"
\end{lstlisting}
\item Образцы: \lstinline|IPv4 образец_String| и \lstinline|IPv6 образец_String|.
\item Пример функции:
\begin{lstlisting}[basicstyle=\ttfamily\small]
isLocalhost (IPv4 "127.0.0.1") = True
isLocalhost (IPv6 "0:0:0:0:0:0:0:1") = True
isLocalhost _ = False
\end{lstlisting}
\pause
\item Ещё пример типа:
\begin{lstlisting}[basicstyle=\ttfamily\small]
data ImageFormat = JPG | PNG | Other String
\end{lstlisting}
\end{itemize}
\end{frame}

\begin{frame}[fragile]
\frametitle{Типы с параметрами (полиморфные)}
\begin{itemize}
    \item Несколько полей одного конструктора соответствуют операции декартова произведения в теории множеств.
    \only<3->{\item В логике конъюнкции, а в алгебре произведению.}
    \item А несколько конструкторов\only<1>{?}\pause\ \enquote{помеченному объединению}.
    \only<3->{\item В логике дизъюнкции, а в алгебре сумме.}
    \item Сравните помеченное объединение с непомеченным \lstinline|union| в C/C++.
    \item И с \lstinline|boost::variant| (\lstinline|std::variant| в C++17).
    \only<3->{\item Тип функций "--- импликация в логике и возведение в степень в алгебре}
    % TODO переделать в таблицу
\end{itemize}
\end{frame}

\begin{frame}[fragile]
\frametitle{Алгебраические типы}
\begin{itemize}
    \item Несколько полей одного конструктора соответствуют операции декартова произведения в теории множеств.
    \only<3->{\item В логике конъюнкции, а в алгебре произведению.}
    \item А несколько конструкторов\only<1>{?}\pause\ \enquote{помеченному объединению}.
    \only<3->{\item В логике дизъюнкции, а в алгебре сумме.}
    \item Сравните помеченное объединение с непомеченным \lstinline|union| в C/C++.
    \item И с \lstinline|boost::variant| (\lstinline|std::variant| в C++17).
    \only<3->{\item Тип функций "--- импликация в логике и возведение в степень в алгебре}
    % TODO переделать в таблицу
\end{itemize}
\end{frame}



%\begin{frame}[fragile]
%\frametitle{Алгебраические типы}
%\begin{itemize}
%    \item Почему эти типы называются алгебраическими? \pause
%    \item Несколько полей одного конструктора соответствуют операции (декартова) произведения. \pause
%    \item Несколько конструкторов типа "--- сумме.
%\end{itemize}
%\end{frame}

\begin{frame}[fragile]
\frametitle{Записи}\hypertarget{rec2}{}
\hyperlink{rec1}{\beamerreturnbutton{К определению записей}}
\begin{lstlisting}[basicstyle=\ttfamily\small]
Prelude> let aPoint = Point 0 (-1)
Prelude> aPoint { xCoord = 1 } !\pause!
Point {xCoord = 1.0, yCoord = -1.0}
Prelude> let Point {xCoord = x'} = it
Prelude> x' !\pause!
1.0 !\pause!
Prelude> :set -XNamedFieldPuns -XRecordWildCards
Prelude> let f (Point { xCoord, yCoord }) = xCoord + yCoord
Prelude> let g (Point { .. }) = x + y
\end{lstlisting}
Оба последних образца "--- сокращения\\ \lstinline[basicstyle=\ttfamily\small]|Point { xCoord = xCoord, yCoord = yCoord }|.
\begin{lstlisting}[basicstyle=\ttfamily\small]
Prelude> Point { xCoord = 0 } !\pause!
\end{lstlisting}
Это компилируется только с предупреждением.
\end{frame}

\end{document}